\documentclass[a4paper]{article}
\usepackage[latin1]{inputenc}
\usepackage[T1]{fontenc}
\usepackage{lmodern}
\usepackage{graphicx}
\usepackage[french]{babel}

\newcommand{\harmonyInHand}{5 }
\newcommand{\boardSize}{10 }
\newcommand{\distanceFromStart}{2 }
\newcommand{\startingDisharmonyDraw}{3 }
\newcommand{\disharmonyDraw}{2 }
\newcommand{\removedDisharmoniesPerPlayer}{1 }

\newcommand{\startingEnergy}{3 }
\newcommand{\maxEnergy}{15 }
\newcommand{\restEnergyRate}{3 }
\newcommand{\feedingEnergyRate}{8 }
\newcommand{\feedingBeverageCost}{2 }
\newcommand{\feedingMealCost}{2 }
\newcommand{\passiveEnergyResplenishmentRate}{1 }

\newcommand{\startingBeverage}{2 }
\newcommand{\maxBeverage}{15 }
\newcommand{\beverageProductionRate}{4 }
\newcommand{\beverageProductionEnergyCost}{3 }

\newcommand{\startingMeal}{1 }
\newcommand{\maxMeal}{15 }
\newcommand{\mealProductionRate}{3 }
\newcommand{\mealProductionEnergyCost}{3 }

\begin{document}

\title{Harmonie}
\author{Pâris MEULEMAN}
\maketitle
\newpage
\tableofcontents
\newpage

\section{Résumé des règles}
\subsection{Mise en place}
	\begin{itemize}
		\item Disposer le plateau principal au centre.
		\item Placer le pion \em{Équilibre} au centre du plateau.
		\item Mélanger le tas \em{Harmonie} et le placer a coté du plateau central.
		\item Préparer le tas \em{Disharmonie}:
		\begin{itemize}
			\item Mélanger chaque étage indépendamment.
			\item Retirer \removedDisharmoniesPerPlayer carte par joueurs en dessous de 5 a chaque étage.
			\item Disposer les étages les un sur les autre dans un ordre décroissant.
		\end{itemize}
		\item Chaque joueur:
		\begin{itemize}
			\item Choisi une couleur.
			\item Place son pion sur le plateau central (sur son élément à \distanceFromStart du centre).
			\item Récupère le plateau individuel et les cartes actions associés a son élément.
			\item Place les marqueurs de ressources sur leur position de départ (\startingEnergy Énergies, \startingBeverage Boissons, \startingMeal Repas).
			\item Pioche \harmonyInHand cartes \em{Harmonies}.
		\end{itemize}
		\item Piocher \startingDisharmonyDraw cartes Disharmonie.
	\end{itemize}
	\subsection{Tour de jeu}
	\begin{itemize}
		\item Chaque joueur choisi une seule action parmi:
		\begin{itemize}
			\item Jouer des cartes \em{Harmonie}.
			\item Remplacer des cartes \em{Harmonie}.
			\item Produire \beverageProductionRate boissons pour un coût de \beverageProductionEnergyCost énergies.
			\item Produire \mealProductionRate repas pour un coût de \mealProductionEnergyCost énergies.
			\item Se reposer pour gagner \restEnergyRate énergies.
			\item Boire \feedingBeverageCost boissons et manger \feedingMealCost pour gagner \feedingEnergyRate énergies.
		\end{itemize}
		\item Chaque joueur exécute l'action choisie.
		\item Chaque joueur regagne \passiveEnergyResplenishmentRate énergies.
		\item Piocher \disharmonyDraw \em{Disharmonies}.
	\end{itemize}
	\subsection{Fin du jeu}
	\begin{itemize}
		\item Si le pion d'un joueur atteins une des extrémités du plateau, il est immédiatement éliminé.
		\item Si le pion \em{Équilibre} atteins une des extrémités du plateau, le monde (et le jeu) est perdu pour tous sauf :
		\item Si le joueur correspondant à l'extrémité atteinte par le pion \em{Équilibre} est encore dans la partie. Auquel cas ce joueur est le seul gagnant. 
		\item S'il n'y a plus de carte \em{Disharmonie}, les joueurs non éliminés ont gagné, chaque joueur marque: un nombre de point égal à sa distance de l'extrémité du plateau la plus proche, ainsi qu'un nombre de point égal à 20 moins sa distance au pion \em{Équilibre}.
	\end{itemize}
\end{document}
		
