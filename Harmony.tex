\documentclass[a4paper]{article}
\usepackage[latin1]{inputenc}
\usepackage[T1]{fontenc}
\usepackage{lmodern}
\usepackage{graphicx}
\usepackage[french]{babel}

\newcommand{\harmonyInHand}{5 }
\newcommand{\boardSize}{10 }
\newcommand{\distanceFromStart}{2 }
\newcommand{\startingDisharmonyDraw}{3 }
\newcommand{\disharmonyDraw}{2 }
\newcommand{\removedDisharmoniesPerPlayer}{1 }

\newcommand{\startingEnergy}{3 }
\newcommand{\maxEnergy}{15 }
\newcommand{\restEnergyRate}{3 }
\newcommand{\feedingEnergyRate}{8 }
\newcommand{\feedingBeverageCost}{2 }
\newcommand{\feedingMealCost}{2 }
\newcommand{\passiveEnergyResplenishmentRate}{1 }

\newcommand{\startingBeverage}{2 }
\newcommand{\maxBeverage}{15 }
\newcommand{\beverageProductionRate}{4 }
\newcommand{\beverageProductionEnergyCost}{3 }

\newcommand{\startingMeal}{1 }
\newcommand{\maxMeal}{15 }
\newcommand{\mealProductionRate}{3 }
\newcommand{\mealProductionEnergyCost}{3 }

\begin{document}

\title{Harmonie}
\author{P�ris MEULEMAN}
\maketitle
\newpage
\tableofcontents
\newpage

\section{R�sum� des r�gles}
\subsection{Mise en place}
	\begin{itemize}
		\item Disposer le plateau principal au centre.
		\item Placer le pion \em{�quilibre} au centre du plateau.
		\item M�langer le tas \em{Harmonie} et le placer a cot� du plateau central.
		\item Pr�parer le tas \em{Disharmonie}:
		\begin{itemize}
			\item M�langer chaque �tage ind�pendamment.
			\item Retirer \removedDisharmoniesPerPlayer carte par joueurs en dessous de 5 a chaque �tage.
			\item Disposer les �tages les un sur les autre dans un ordre d�croissant.
		\end{itemize}
		\item Chaque joueur:
		\begin{itemize}
			\item Choisi une couleur.
			\item Place son pion sur le plateau central (sur son �l�ment � \distanceFromStart du centre).
			\item R�cup�re le plateau individuel et les cartes actions associ�s a son �l�ment.
			\item Place les marqueurs de ressources sur leur position de d�part (\startingEnergy �nergies, \startingBeverage Boissons, \startingMeal Repas).
			\item Pioche \harmonyInHand cartes \em{Harmonies}.
		\end{itemize}
		\item Piocher \startingDisharmonyDraw cartes Disharmonie.
	\end{itemize}
	\subsection{Tour de jeu}
	\begin{itemize}
		\item Chaque joueur choisi une seule action parmi:
		\begin{itemize}
			\item Jouer des cartes \em{Harmonie}.
			\item Remplacer des cartes \em{Harmonie}.
			\item Produire \beverageProductionRate boissons pour un co�t de \beverageProductionEnergyCost �nergies.
			\item Produire \mealProductionRate repas pour un co�t de \mealProductionEnergyCost �nergies.
			\item Se reposer pour gagner \restEnergyRate �nergies.
			\item Boire \feedingBeverageCost boissons et manger \feedingMealCost pour gagner \feedingEnergyRate �nergies.
		\end{itemize}
		\item Chaque joueur ex�cute l'action choisie.
		\item Chaque joueur regagne \passiveEnergyResplenishmentRate �nergies.
		\item Piocher \disharmonyDraw \em{Disharmonies}.
	\end{itemize}
	\subsection{Fin du jeu}
	\begin{itemize}
		\item Si le pion d'un joueur atteins une des extr�mit�s du plateau, il est imm�diatement �limin�.
		\item Si le pion \em{�quilibre} atteins une des extr�mit�s du plateau, le monde (et le jeu) est perdu pour tous.
		\item S'il n'y a plus de carte \em{Disharmonie}, les joueurs non �limin�s ont gagn�, chaque joueur marque: un nombre de point �gal � sa distance de l'extr�mit� du plateau la plus proche, ainsi qu'un nombre de point �gal � 20 moins sa distance au pion \em{�quilibre}.
	\end{itemize}
\end{document}
		